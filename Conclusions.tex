% Chapter 6

\chapter{Conclusions} % Main chapter title

\begin{figure}[hbtp]

\centering
    \includegraphics[width=0.7\textwidth]{results/v2all.jpg}
    \rule{35em}{0.5pt}
    \caption[Elliptic Flow vs Transverse Momentum, 200 GeV d+Au]{Elliptic Flow vs Transverse Momentum, 200 GeV d+Au}
    \label{fig:v2main}
\end{figure}

Measurements of the second Fourier coefficient corresponding to an elliptically shaped azimuthal anisotropy of pions, kaons, and (anti)protons produced in 200 GeV deuteron-gold collisions are presented in figure \ref{fig:v2main}. Historically, a large number of participant nucleons was thought to be required to be needed in order to produce a QGP. Consequently, it was thought that the low number of participant nucleons in the d+Au system was insufficient for such a phase change. The nonzero flow measurement in this analysis is a sign that collective behavior happens in systems previously thought of as ``cold'' and is an indication that a QGP could be formed in the simpler system of d+Au. This flow increases steadily for all hadrons up to $p_T \sim 1.5 $ GeV/c where the mesons (pions and kaons) seem to reach a saturation and flatten out. The kaons exhibit a flow signal stronger than the pions in this range but eventually decrease to the same nominal value as the pions. The (anti)protons continue to flow increasingly up to $p_T \sim 2$ GeV/c where they too flatten out. The measurement of particles and their corresponding antiparticles was separated throughout the course of the analysis. In doing so, flow coefficients for negative and positive charged particles for a given species can be compared. Here we see that the collision, evolution, and freeze out of the d+Au system does not favor particles or antiparticles, as expected.

\section{Discussion}

\subsection{Hadronization}
\begin{figure}[hbtp]
\centering    
    \includegraphics[width=0.7\textwidth]{results/v2NqvspT.jpg}
    \rule{35em}{0.5pt}
    \caption[Quark Scaled Elliptic Flow ($\pi^{\pm}$ and $p/\bar{p}$) vs Transverse Momentum, 200 GeV d+Au]{Quark Scaled Elliptic Flow ($\pi^{\pm}$ and $p/\bar{p}$) vs Transverse Momentum, 200 GeV d+Au}
    \label{fig:qscaledv2}
\end{figure}

The (anti)proton flow enhancement in the mid-$p_T$ range is similar to the baryon enhancement seen in previous experiments and the observation of collective flow is a strong indication of quark deconfinement/QGP formation. Because of this deconfinement, baryon enhancement must happen with some mechanism in the freeze out stage. The leading model describing this phenomenon, recombination, can be seen by scaling the flow coefficient and momentum by the number of quarks that comprise the measured particle\footnote{2 quarks for pions/kaons, 3 quarks for protons}. Doing so shows that this particle momentum discrepancy may come from the sum of momenta of the constituent quarks. That is to say, the reason why protons appear to have stronger flow is simply because they contain more quarks. For example, if we were to take three deconfined quarks that have the same momentum, they would hadronize to produce a particle with higher momentum than if only two of those quarks had combined. This quark scaled flow is shown in figure \ref{fig:qscaledv2} for pions and protons and shows that both particle signals increase and reach saturation in the same range. That the two quark scaled measurements track each other well through the $p_T$ range is a strong indication of recombination as the mechanism for thermal freeze-out and is a result that agrees with similar quark scaled results in $^3$He+Au\citep{huangQM2015}, Au+Au\citep{Adler:2003kt}, Cu+Cu\citep{PhysRevC.92.034913}, Pb+Pb\citep{Noferini:2012ps} systems both at RHIC and abroad.

\subsection{Comparison to Flow Models}
\begin{figure}[hbtp]
\centering    
    \includegraphics[width=0.5\textwidth]{results/v2allpipmodels.jpg}
    \rule{35em}{0.5pt}
    \caption[$\pi^{\pm}$ and $p/\bar{p}$ Elliptic Flow compared to hydrodynamic models.]{$\pi^{\pm}$ and $p/\bar{p}$ Elliptic Flow compared to hydrodynamic models: MC-Glauber\citep{Nagle:2013lja}, IP-Glasma\citep{Schenke:2014gaa}, and superSONIC\citep{Romatschke2015}}.
    \label{fig:hydrov2}
\end{figure}

Prior to hadronization, the equilibrated state of deconfined quarks has been described well with viscous hydrodynamic models. The way these models differ is in their assumption of the initial conditions (Glauber versus CGC) and length of equilibration time. A comparison of this measurement with these hydrodynamic models is shown in figure \ref{fig:hydrov2}. Models using Monte Carlo simulations of identified particle flow that utilize Glauber initial conditions (MC-Glauber) describe the data well for low $p_T$ but diverge above $p_T \sim 1$ GeV/c, also seen in d+Au in a previous PHENIX analysis\citep{Adare:2014keg}\footnote{though diverging at slightly higher transverse momentum, $p_T < 1.5 $GeV/c}. A continuation of MC-Glauber including a longer equilibration time (longer period of \textit{pre-flow} before a thermalized QGP flow) and the effect of post freeze-out hadron interaction\footnote{referred to in literature as \textit{Hadronic Cascade Afterburner}} called \textit{superSONIC}\citep{Romatschke2015}\footnote{an extension of an earlier SONIC model with a longer pre-flow time. SONIC stands for \textit{Super hybrid mOdel simulatioN for relativistic heavy-Ion Collisions}\citep{Romatschke2015}} matches data through mid $p_T$ but does not match the flow signal's saturation that begins just below $p_T \sim 2$ GeV/c, an effect also seen in the aformentioned PHENIX d+Au analysis. A model using an impact parameter independent CGC initial condition with a Glasma thermalization phase (IP-Glasma)\citep{Schenke:2014gaa} overestimates the flow strength at low $p_T$ but does appear to model the asymptotic flow behavior at mid-high $p_T$, albeit overestimating the the value of that maximum flow value slightly. These two models can be seen compared to the overall flow in figure \ref{fig:allhadronhydro}. In order to minimize the effect of baryon enhancement in mid $p_T$, the elliptic flow of all hadrons is approximated by summing the yields of all identified tracks and performing a flow analysis. 

\begin{figure}[hbtp]
\centering    
    \includegraphics[width=0.5\textwidth]{results/v2hydro.jpg}
    \rule{35em}{0.5pt}
    \caption[Elliptic flow of all charged hadrons compared to hydrodynamic models.]{Elliptic flow of all charged hadrons measured by summing the yields of all identified tracks. Data is compared to IP-Glasma and superSONIC models.}
    \label{fig:allhadronhydro}
\end{figure}

\subsection{Strangeness and Initial Conditions}
While quark scaling can be used to explain pion and proton flow, implying that these hadrons are made by the recombination of light flavor quarks, these quark building blocks had already existed before the collision in the form of valence quarks in the ions. What about strange quarks? These are created fresh from the collision, be it because of initial conditions\footnote{gluon saturation from CGC}, because of some process in thermalization\footnote{Glasma/pre-equilibrium flow}, or because of some strange quark producing mechanism in the flowing QGP\footnote{gluon fusion}. Previous enhancement was explained with a hadronization mechanism made evident by quark scaling protons and pions. However this does not work for kaons as shown in figure \ref{fig:qscalemesonflow}. Here the quark scaled the kaon flow does not follow the first generation hadrons' flow in the range $p_T \sim 0.7-1.3$ GeV/c, rather, it is stronger.

\begin{figure}[hbtp]
\centering
    \begin{subfigure}[h]{0.48\textwidth}
    \centering
    \includegraphics[width=1\textwidth]{results/v2mesons.jpg}

    \caption{Meson flow ($\pi^{\pm}$ and $k^{\pm}$).}
    \label{fig:smesonflow}
	\end{subfigure}
    \begin{subfigure}[h]{0.48\textwidth}
    \centering
    \includegraphics[width=1\textwidth]{results/v2NqvspTwmodels.jpg}
    \caption{Quark scaled elliptic flow ($\pi^{\pm}$,$k^{\pm}$,$p/\bar{p}$).}
    \label{fig:qscalemesonflow}

	\end{subfigure}
	\begin{subfigure}[h]{0.48\textwidth}
    \centering
    \includegraphics[width=1\textwidth]{results/v2mesonratio.jpg}
    \caption{Meson flow ratio ($v_2^{k} / v_2^{\pi}$).}
    \label{fig:mesonratio}

	\end{subfigure}

    \rule{35em}{0.5pt}
    \caption[Meson Elliptic Flow plots]{Elliptic flow of mesons ($\pi^{\pm}$ and $k^{\pm}$) are compared to hydrodynamic models. Where hadrons comprised from first generation quarks obey quark scaling, strange quark-containing kaons show an excess. CGC containing model seems to model kaon behavior at mid-$p_T$ well.}
    \label{fig:qscaledhydro}
\end{figure}

In this range, the IP-Glasma model best fits the kaon flow (see fig. \ref{fig:smesonflow}) implying that the choice of initial conditions and thermalization mechanisms in models may explain strangeness enhancement and may be an indication of how CGC/Glasma may present itself in simple heavy ion systems. Following this thought, the Glauber model would be appropriate for describing first generation quark flow since it treats the nucleus as a density function of these first generation quarks. On the other hand, the CGC initial condition has built within it a plethora of gluons due to low-x saturation. Strangeness enhancement through gluon-gluon fusion has historically been proposed as a sign of the onset of QGP formation\citep{PhysRevLett.48.1066}, a phenomena that combines well with the dominant availability of gluons in the CGC model. Similar strange flow enhancement above quark scaling has been seen in\footnote{in triangular flow ($v_3$) measured with $k^{\pm}$ and $\Lambda/\bar{\Lambda}$ compared to $\pi^{\pm}$ and $p/\bar{p}$.} Pb+Pb collisions\citep{1742-6596-668-1-012099} and at high kinetic energy in $^3$He+Au\citep{huangQM2015}. However, previous Au+Au results have shown that kaons follow quark scaling\citep{PhysRevC.92.034913}. 

As a thought experiment to explain why strangeness enhancement is observed in some systems but not others, consider other suppressions observed in Au+Au. Heavy flavor suppression from the QGP has been observed with charm quarks in $J/\psi$, it may be that a similar suppression happens with strange quarks making the kaon flow appear to scale with the number of quarks in Au+Au systems. That is to say, kaon enhancement may due to the CGC initial condition but are then suppressed due to heavy flavor suppression in the QGP and in Au+Au the combination of the two makes the kaons appear to scale with the number of quarks. Furthermore, STAR results show that $J/\psi$ suppression disappears in d+Au\citep{Powell:2010oea} which would provide motivation for why the strangeness enhancement would dominate in this system. The observation of kaon enhancement in Pb+Pb may imply that this QGP heavy flavor suppression only happens for a finite range of $p_T$ or for a sufficiently low system energy density which may be supported by ALICE's observation of $J/\psi$ elliptic flow in 2.76 TeV Pb+Pb collisions\citep{ALICE:2013xna}.

\section{Summary}
There is ample evidence from this analysis to say that there is collective flow in the previously thought ``cold'' system of d+Au. Quark scaling proton and pion measurements point to a likely recombination model mechanism for first generation quark hadronization, and the flow of these hadrons is modeled well with viscous hydrodynamics with Glauber model initial conditions. There are many questions to be answered regarding strangeness. Since nuclei are made of light flavor quarks, treating their distribution as a density distribution as Glauber does is perfectly adequate, however it does not model the availability of strange quarks with which partons may scatter and production processes such as gluon fusion which which strange quarks can be formed. Strange hadrons may also provide a way to better study the initial conditions\footnote{Color-Glass Condensate} and thermalization\footnote{Glasma/preflow} of the QGP. 

In closing, this observation of collective phenomena in simple systems is evidence that there is always something new to be learned from things already studied. As we gaze ever further into the depths of space, ever closer into the intricacies of matter, we are met with increasing profundity. Things we thought we knew we question and things we thought not possible become likely. It is an exciting time as we continue to make discoveries in all subfields of physics and, while we search for the things current theories predict, some of the most exciting discoveries to come will likely present themselves in ways we never conceived and in places we never thought to look. 

\pagebreak
\pagebreak


