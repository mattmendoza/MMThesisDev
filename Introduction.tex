\chapter{Introduction}

% Chapter 1

\section{Introduction}
%----------------------------------------------------------------------------------------
Humanity is set apart from the rest of creation by its tenacity to explore and discover and it's no surprise that the driving force for this discovery is our constant desire to better understand the universe around us. From the deepest distances of outer space to the infinitesimally small structures of the atoms that comprise the matter we interact with everyday, our growing understanding of the intricacies of our world has revealed such amazing complexities we never could have thought possible. Over the last century, one such exploration that has uncovered many unforeseen phenomena is the study of the building blocks of matter and the forces that hold it together. From Planck's discovery that the world of the incredibly small is not smooth and continuous but rather comprised of discrete quanta, extending to Bohr's model of the atom showing that even the orbital dynamics of electrons were non classical, to the discovery of the neutron leading the way to the discovery of new fundamental forces of nature, field theories to describe them, and eventually the venerable standard model of particle physics, our ever increasing understanding of the workings of the atom has revealed many discoveries that were surprising to those who sought them. In the last few decades one such investigation into the makings of the atomic nucleus seeks to study the properties of this nuclear material under temperature and density extrema, namely the search for phenomena where quarks and gluons that make up a solid phase of nuclear matter like protons and neutrons in a nucleus would become deconfined and melt into a novel state of matter called the \textit{Quark Gluon Plasma} (QGP) at said extrema. 
First discussions of this deconfinement were found when Collins and Perry \citep{Collins:1974ky} said that at small distances the coupling constant would decrease implying that dense nuclear matter would comprise of deconfined quarks and gluons. Though their inferences pertained to low temperature, high density systems, they noted that similar phenomena could occur in high temperature systems like that of the early universe. The first to coin the term "Quark-Gluon Plasma" was Shuryak in 1980 \citep{Shuryak:1980tp} who wrote: 

\begin{addmargin}[1.5em]{2em}
"When the energy density $\epsilon$ exceeds some typical hadronic value ($\sim$1 GeV/fm$^{3}$), matter no longer consists of separate hadrons (protons, neutrons, etc.), but as their fundamental constituents, quarks and gluons. Because of the apparent analogy with similar phenomena in atomic physics we may call this phase of matter the QCD (or quark–gluon) plasma."
\end{addmargin}
 
With this theoretical framework, we have set out to develop an understanding of this phase transition whose resulting deconfinement of quarks and gluons has led to a new era of physics discoveries.

\section{Early Experiments: An Overview}
The earliest experiments that utilized collisions of two ions to study nuclear matter were largely re-purposing existing accelerators that were used for elementary particle physics. Whereas the general goal of an accelerator setup to study elementary particle physics is to study the production of exotic single particles created in specific Quantum Field Theory processes (baryons, mesons, force carrier bosons, leptons) an accelerator used for Heavy Ion Physics is used to study larger nuclear matter systems which result in higher track multiplicity consisting of particles created by more common processes. 

Examples of this include the re-purposing of the weak focusing proton synchrotron called the Bevatron at Lawrence Berkeley National Laboratory when it was joined with the SuperHiLac, a linear accelerator capable of accelerating ions to relativistic energies of up to 2 GeV per nucleon and became the only machine in the world capable of accelerating all of the elements in the periodic table to relativistic speeds. This capability allowed researchers to pioneer the study of "quark matter" \citep{bevalac9lives} the crowning achievement of this epoch being the discovery that nuclear matter could be compressed to high temperature \citep{ROBINSON857}, made evident by the observation of collective flow in Niobium + Niobium ions at 400 MeV/nucleon \citep{PhysRevLett.52.1590} which paved the way for the search for other phases of nuclear matter, namely the Quark Gluon Plasma.

The first step into the "ultra-relativistic" energy regime ($>$ 10 GeV/nucleon) took place at Brookhaven National Lab in the mid 80's with the Alternating Gradient Synchrotron which initially was able to reach 14 GeV per nucleon with Silicon ions. Concurrently across the ocean, the European Organization for Nuclear Research (Conseil Européen pour la Recherche Nucléaire, CERN) had the Super Proton Synchrotron accelerated Oxygen and Sulfur ions up to 200 GeV per nucleon. By the mid 90's both had seen their own upgrades that allowed them to create larger systems with the use of so called "Heavy" ions. At the AGS, Gold ions accelerated to 11 GeV per nucleon, and at the SPS, Lead ions accelerated to 158 GeV per nucleon\citep{wojciechphenomenology}. Both accelerators set the stage for RHIC, each with their own prolific periods of discovery. 

At the AGS, the observation of directed and elliptic flow in 11.5 GeV Au + Au collisions at the AGS by the E877 collaboration \citep{Barrette:1999rx} showed that quark matter behaved collectively, that it was indeed a new state of matter not just a conglomerate of independent nuclei. The E802 collaboration compared the production of charged kaons compared to charged pions in 14.6 GeV Si + Au collisions and saw the "strange" result that the production of kaons was enhanced compared to proton + proton collisions showing that the collective behavior of this quark matter produced a different spectrum of particles than the simpler p + p collisions \citep{PhysRevLett.64.847}. 

At the SPS, this strangeness enhancement was also studied by the NA49\citep{Hohne:1999jf} collaboration through a similar measurement of the charged kaon to pion ratio and the WA97\citep{Antinori:1999hy} collaboration measuring the production of multi-strange baryons ($\Lambda$,$\bar{\Lambda}$, $\Xi^{\pm}$, and $\Omega^{\pm}$). Meanwhile the NA50 collaboration was finding not enhancement but suppression of J/$\Psi$ mesons (which decay to $l^{+}+l^{-}$ pairs) when compared to $l^{+}+l^{-}$ pairs created through Drell-Yan processes (see fig \ref{fig:DYfeynman}) \citep{Abreu:2000ni}.

Concurrently, the teams at the two accelerators used two particle correlations to study the evolving collision volume and saw that the energy density of the quark matter was considerably higher, an order of magnitude greater, than naturally stable laboratory nuclei \citep{Heinz:1999rw} and found the conditions for thermal freeze out \citep{BraunMunzinger:1998cg}, i.e. the re-confinement of quarks and gluons into hadronic states, a process called \textit{hadronization}.

These various phenomena were signatures of new physics that could not be explained by solely scaling up $p+p$ collisions, they were signs that the nuclear material had undergone a transformation, more specifically, a phase transition and a sign of quark-gluon deconfinement. 
\pagebreak
\pagebreak
