\chapter{Introduction}

% Chapter 1

\section{Introduction}
%----------------------------------------------------------------------------------------
Humanity is set apart from the rest of creation by its tenacity to explore and discover and it's no surprise that the driving force for this discovery is our constant desire to better understand the universe around us. From the deepest distances of outer space to the infinitesimally small structures of the atoms that comprise the matter we interact with everyday, our growing understanding of the intricacies of our world has revealed such amazing complexities we never could have thought possible. Over the last century, one such exploration that has uncovered many unforeseen phenomena is the that of the building blocks of matter and the forces that hold it all together. From Planck's discovery that the world of the incredibly small is not smooth and continuous but rather comprised of discrete quanta, to Bohr's model of the atom showing that even the orbits of electrons were quantized, a new quantum world changed the very way we approached physics. This was further solidified with the discovery of the neutron which pointed the way to the discovery of new fundamental forces of nature, field theories with which to describe them, and eventually the venerable standard model of particle physics. At every turn, our ever increasing understanding of the workings of the atom revealed many new surprises. In the last few decades, one such investigation into the makings of the atomic nucleus aims to study the properties of this nuclear material under extremes of temperature and energy density. As we now know, the proton and neutron were not the proverbial 
``end-of-the-line'' for physicists seeking the fundamental components of matter. We discovered that they, too, are comprised of smaller particles called \textit{quarks}, which are held together with a new fundamental force called the strong force, which is mediated by the exchange of force carrier particles called \textit{gluons.} This binding of quarks into confined states such as protons and neutrons made physicists wonder about the nature of this confinement, namely whether or not it was possible to deconfine particles into their constituent quarks. The 2004 Nobel Prize in Physics went to Gross, Wilczek\citep{PhysRevD.8.3633}, and Politzer\citep{PhysRevLett.30.1346} for their 1973 discovery of this asymptotic freedom in the quantum field theory of the strong nuclear force, Quantum Chromodynamics (QCD). They discovered that the strength of the strong force became asymptotically weaker as energy increased and distance decreased. Collins and Perry\citep{Collins:1974ky} continued this idea to the nuclear matter extremes that exist in the centers of neutron stars and exploding black holes, noting that, due to the extreme pressures in these systems resulting in small distances between quarks, the QCD coupling constant would decrease resulting in asymptotically free quarks. Though their inferences pertained to low temperature, high density systems, they noted that similar phenomena could occur in high temperature systems like that of the early universe. The first to coin the term ``Quark-Gluon Plasma'' was Shuryak in 1980 \citep{Shuryak:1980tp} who wrote: 

\begin{addmargin}[1.5em]{2em}``When the energy density, $\epsilon$, exceeds some typical hadronic value ($\sim$1 GeV/fm$^{3}$)\footnote{this energy density was later found to be $\epsilon \approx 2$ GeV/fm$^3$ \citep{Fries:2006pv}}, matter no longer consists of separate hadrons (protons, neutrons, etc.), but as their fundamental constituents, quarks and gluons. Because of the apparent analogy with similar phenomena in atomic physics, we may call this phase of matter: the QGP (or quark gluon) plasma.''
\end{addmargin}
 
It is with this theoretical framework that we have set out to develop an understanding of this quark/gluon deconfinement. But what of this Quark-Gluon Plasma? How is it created, and what are its properties? Historically, new physics discoveries have led to new frontiers of science. Could this search for the QGP deepen our understanding of nuclear forces beyond small system interactions to that of larger systems? And could a better understanding of these larger systems point to ever stranger new phenomena? It was questions such as these that have led us to a new era of physics discovery.

\section{Early Experiments: An Overview} \label{sect:earlyexperiments}
The earliest experiments that utilized the collisions of two ions to study nuclear matter were largely the re-purposing existing accelerators that were used for elementary particle physics. Whereas the general goal of an accelerator setup to study elementary particle physics is to study the production of new particles by the collision of small nucleons or leptons that result in a small number of detected particle tracks, an accelerator used for Heavy Ion Physics is used to study larger nuclear matter systems created by colliding large nuclei which results in higher track multiplicity consisting of particles created by more common QCD processes. 

Examples of this include the re-purposing of the weak focusing proton synchrotron called the Bevatron at Lawrence Berkeley National Laboratory when it was joined with the SuperHiLac, a linear accelerator capable of accelerating ions to relativistic energies of up to 2 GeV per nucleon and became the only machine in the world at the time capable of accelerating all of the elements in the periodic table to relativistic speeds. This capability allowed researchers to pioneer the study of \textit{quark matter}\citep{bevalac9lives}; the major achievement of this epoch being the discovery that nuclear matter could be compressed to high temperatures \citep{ROBINSON857}. This property was made evident by the observation of collective flow in Niobium + Niobium collisions that were accelerated to energies of 400 MeV/nucleon \citep{PhysRevLett.52.1590}, paving the way for the search for other phases of nuclear matter, namely the Quark Gluon Plasma.

The first step into the ``ultra-relativistic'' energy regime ($>$ 10 GeV/nucleon) took place at Brookhaven National Lab (BNL) in the mid 80's with the \textit{Alternating Gradient Synchrotron} (AGS) which initially was able to reach 14 GeV per nucleon with Silicon ions. Concurrently across the ocean, the European Organization for Nuclear Research (Conseil Européen pour la Recherche Nucl\'{e}aire, CERN) had the \textit{Super Proton Synchrotron} (SPS) which accelerated Oxygen and Sulfur ions up to 200 GeV per nucleon. By the mid 90's both had seen their own upgrades that allowed them to create larger systems with the use of so called ``Heavy'' ions. At the AGS, gold ions were accelerated to 11 GeV per nucleon, and at the SPS, lead ions were accelerated to 158 GeV per nucleon\citep{wojciechphenomenology}. Though both accelerators set the stage for the \textit{Relativistic Heavy Ion Collider}, each had their own periods of discovery. 

The first sign that quark matter behaved collectively was the observation that it ``flowed'' like a fluid would. This phenomena, called \textit{collective flow}, was observed in 11.5 GeV Au + Au collisions at the AGS by the E877 collaboration \citep{Barrette:1999rx}. This quark matter was indeed a new state of matter as it did not behave simplya as a conglomerate of independent nuclei. The E802 collaboration compared the production of charged kaons compared to charged pions in 14.6 GeV Si + Au collisions and saw that the production of kaons was enhanced compared to proton + proton collisions showing that the collective behavior of this quark matter produced a different spectrum of particles than the simpler p + p collisions \citep{PhysRevLett.64.847}. 

At the SPS, this strangeness enhancement was also studied by the NA49\citep{Hohne:1999jf} and WA97 collaborations through the measurement of the charged kaon to pion ratio and the production of multi-strange baryons ($\Lambda$,$\bar{\Lambda}$, $\Xi^{\pm}$, and $\Omega^{\pm}$). Meanwhile the NA50 collaboration was finding not enhancement but rather the suppression of J/$\Psi$ mesons in heavy ion collisions compared that in p+p collisions\citep{Abreu:2000ni}. This was caused by quarks and gluons at high temperature causing a color charge screening effect on the charmed quark pair that comprised the J/$\Psi$, breaking up the charm pair that made up the J/$\Psi$.

Concurrently, the teams at both accelerators used two particle correlations to study the evolving collision volume and saw that the energy density of the quark matter was considerably higher, a factor of 10 greater, than naturally stable laboratory nuclei \citep{Heinz:1999rw} and found the lower boundary of QGP formation by finding the conditions (energy density $\leq$ 100 MeV/fm$^{3}$) for thermal \textit{freeze-out}\citep{BraunMunzinger:1998cg}, i.e. the point of re-confinement for quarks and gluons into hadronic states, a process called \textit{hadronization}.

These various phenomena were signatures of new physics that could not be explained by solely scaling up $p+p$ collisions. They were signs that new undiscovered mechanisms were at work when large numbers of nucleons came together such as in a nucleus or in extreme conditions such as those in the early universe and inside of ultra-dense astrophysical objects. 
\pagebreak
\pagebreak
