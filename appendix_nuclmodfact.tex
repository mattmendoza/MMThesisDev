\chapter{Nuclear Modification Factor} % Main appendix title
\label{nuclmodfactapp}

\section{Nuclear Modification Factor}
Since heavy ion systems are comprised of many nucleons colliding, we would like to differentiate between phenomena from singular nucleon-nucleon collisions and phenomena created from interactions of nuclear matter. It is therefore convenient to define a quantity that describes how different a system of N colliding nucleons differs from a system of two colliding nucleons scaled up by N, often referred to as \textit{Binary Scaling}. For example, consider the system created with Au+Au collisions. If we wanted to know how pion production was affected by nuclear matter we would be interested in how different pion production was in the collision of 197 nucleons with another 197 nucleons compared to pion production in p+p scaled up by a factor of 197. This quantity is called a \textit{Nuclear Modification Factor} and is usually denoted with the letter R and two subscripts defining what kind of Nuclear Modification Factor it is:

\begin{equation}
R_{AA}(p_{T}) = \frac{1}{\langle N_coll \rangle} \frac{dN_{AA}/dp_{T}}{d\sigma_{pp}/dp_{T}},
\end{equation}
where $dN_{AA}/dp_{T}$ is the differential yield in the system, $d\sigma_{pp}/dp_{T}$ is the differential cross section in proton-proton collisions, and $\langle N_coll \rangle$ is the number of binary collisions. Given this definition if $R_{AA}=1$ then we expect no new phenomena in the system, that is, the system comprised of N colliding nucleons behaves exactly as we'd expect N proton-proton collisions to behave. If $R_{AA}>1$ it is said that interactions with the nuclear material enhances production and vice versa with $R_{AA}<1$, production is suppressed.