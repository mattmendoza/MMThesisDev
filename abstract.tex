\begin{abstract}
Collective flow has historically been an indicator that nuclear matter created in heavy ion collisions has undergone a phase change to a novel state where its constituent particles are deconfined. This phase called a \textit{Quark-Gluon Plasma} (QGP) has many characteristics that are signature of its creation. Chief among these is the collective behavior of the nuclear matter indicated by its anisotropic flow, as well as high $p_T$ particle suppression, baryon enhancement at mid-$p_T$, and the enhancement of strange quark containing particles above binary scaling expectations. Recent results from CMS at the LHC show evidence of collective flow in the simpler p+Pb system, implying that a QGP may be formed in smaller systems than previously thought. An elliptic flow measurement with identified particles in d+Au collisions could reveal more about the nuclear matter created in these simpler systems. The Pioneering High Energy Nuclear Ion Experiment, or PHENIX, Time of Flight detector used in conjunction with its Aerogel Cherenkov Counter can provide particle identification with good proton/kaon/pion separation for $p_T<7$ GeV/c. I will present studies of particle identified elliptic flow using these detectors which could help to elucidate the underlying physics of the baryon excess and strangeness enhancement anomalies that may be evidence of new physics in the d+Au system.
\end{abstract}